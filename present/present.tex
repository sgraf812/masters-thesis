\documentclass{beamer}
\usetheme[englisch]{KIT}

\usepackage[utf8]{inputenc}
\usepackage[T1]{fontenc}
\usepackage{babel}
\usepackage{booktabs}
\usepackage{tikz,calc,ifthen}
\usepackage{mathtools}
\usepackage[normalem]{ulem}
\usepackage{graphicx}
\usepackage{xspace}
\usepackage{xcolor}
\usepackage{minted}
\usepackage{realboxes}
\usepackage{pifont}
\usepackage{stmaryrd}
\usepackage[maxcitenames=2,backend=bibtex8,hyperref=true,style=authoryear]{biblatex}
\usepackage{hyperref}

\hypersetup{
  pdfdisplaydoctitle=true,
  %colorlinks=true, % use this if the boxes around links bother you ;)
  hidelinks=true,
  pdfstartpage=3,
}
\usepackage{hypcap} % hyperref fix
\usepackage[nameinlink,noabbrev,capitalise]{cleveref} % better references

\bibliography{bib}

\usetikzlibrary{positioning,calc,arrows,shapes}
\tikzset{
  every node/.style={transform shape},
  auto,
  block/.style={align=center,rectangle,draw,minimum height=20pt,minimum width=30pt},
  >=triangle 60,
  alt/.code args={<#1>#2#3}{%
      \alt<#1>{\pgfkeysalso{#2}}{\pgfkeysalso{#3}}
  },
  beameralert/.style={alt=<#1>{color=green!80!black}{}},
  mythick/.style={line width=1.4pt}
}

\newcommand*{\maxwidthofm}[2]{\maxof{\widthof{$#1$}}{\widthof{$#2$}}}
\newcommand<>*{\robustaltm}[2]{
  \alt#3
  {\mathmakebox[\maxwidthofm{#1}{#2}]{#1}\vphantom{#1#2}}
    {\mathmakebox[\maxwidthofm{#1}{#2}]{#2}\vphantom{#1#2}}
}

\definecolor{myred}{HTML}{cc4125}
\definecolor{mygreen}{HTML}{6aa84f}
%\definecolor{myblue}{HTML}{42A1F4}
\definecolor{myblue}{HTML}{75A3CB}

\newcommand<>*{\nodealert}[1]{\only#2{\draw[overlay,mythick,color=green!80!black] (#1.north west) rectangle (#1.south east)}}

%Stuff

\newcommand{\cmark}{\ding{51}}%
\newcommand{\xmark}{\ding{55}}%

% Benchmarks

\newcommand*{\varfull}{\textbf{(1)}\xspace}
\newcommand*{\varedges}{\textbf{(2)}\xspace}
\newcommand{\progname}[1]{\texttt{#1}}
\newcommand{\totalname}[1]{#1}
\newcommand{\andmore}[1]{\emph{... and #1 more}}

%%% Typographic defs

% From https://stackoverflow.com/a/39363004/388010
% Add a period to the end of an abbreviation unless there's one
% already, then \xspace.
\makeatletter
\DeclareRobustCommand\onedot{\futurelet\@let@token\@onedot}
\def\@onedot{\ifx\@let@token.\else.\null\fi\xspace}

\def\eg{e.g\onedot} \def\Eg{E.g\onedot}
\def\ie{i.e\onedot} \def\Ie{I.e\onedot}
\def\cf{c.f\onedot} \def\Cf{C.f\onedot}
\def\etc{etc\onedot} \def\vs{vs\onedot}
\def\wrt{w.r.t\onedot} \def\dof{d.o.f\onedot}
\def\etal{et al\onedot}
\makeatother

\newcommand{\newdef}[1]{\emph{#1}}

\newcommand{\anal}[1]{\textsc{#1}\xspace}
\newcommand*{\MayAlias}{\anal{MayAlias}}
\newcommand*{\MustAlias}{\anal{MustAlias}}
\newcommand*{\MinCard}{\anal{MinCard}}
\newcommand*{\MaxCard}{\anal{MaxCard}}

%%% Minted for syntax highlighting

\newmintinline[hs]{haskell}{}
\newminted[haskell]{haskell}{linenos,mathescape,texcomments,autogobble}

% Syntax
\newcommand{\keyword}[1]{\text{\textsf{\textbf{#1}}}}
\newcommand{\sMkPair}{\text{\textsf{(,)}}}
\newcommand{\sApp}[2]{#1\,#2}
\newcommand{\sLam}[2]{\lambda{}#1.\,#2}
\newcommand{\sLet}[2]{\keyword{let}~#1~\keyword{in}~#2}
\newcommand{\sLetRec}[2]{\keyword{letrec}~#1~\keyword{in}~#2}
\newcommand{\sPair}[2]{(#1,#2)}
\newcommand{\sCase}[4]{\keyword{case}~#1~\keyword{of}~\sPair{#2}{#3}\rightarrow#4}
\newcommand{\sITE}[3]{\keyword{if}~#1~\keyword{then}~{#2}~\keyword{else}~{#3}}

\newcommand{\bind}{x_1=e_1}
\newcommand{\binds}{\overline{x_i=e_i}}

% Co-call Graphs
\newcommand{\edge}[2]{#1\text{---}#2}
\newcommand{\neighbors}[2]{N_{#1} ({#2})}

% Semantic sets
\newcommand{\sVar}{\text{\textsf{Var}}}
\newcommand{\sTrue}{\text{\textsf{True}}}
\newcommand{\sFalse}{\text{\textsf{False}}}
\newcommand{\sExp}{\text{\textsf{Exp}}}
\newcommand{\sVal}{\text{\textsf{Val}}}
\newcommand{\sGraph}{\text{\textsf{Graph}}}
\newcommand{\sUse}{\text{\textsf{Use}}}
\newcommand{\sUsage}{\text{\textsf{Usage}}}
\newcommand{\sMulti}{\text{\textsf{Multi}}}
\newcommand{\sSig}{\text{\textsf{Sig}}}
\newcommand{\sUseEnv}{\text{\textsf{UseEnv}}}
\newcommand{\sUType}{\text{\textsf{UType}}}
\newcommand{\sUTrans}{\text{\textsf{UTrans}}}
\newcommand{\sTransEnv}{\text{\textsf{TransEnv}}}

% Lattice

\newcommand{\llub}{\sqcup}
\newcommand{\lbiglub}{\bigsqcup}
\newcommand{\lleq}{\sqsubseteq}
\newcommand{\lless}{\sqsubset}
\newcommand{\lPair}[2]{\left\langle{}#1,#2\right\rangle{}}
\newcommand{\lTriple}[3]{\left\langle{}#1,#2,#3\right\rangle{}}
\newcommand{\both}{\,\&\,}
\newcommand{\mto}{\to_{+}}

% Transfer function
\newcommand{\transfer}[2]{\mathcal{T}\llbracket#1\rrbracket_{#2}}
\newcommand*{\letupsc}{\textsc{LetUp}\xspace}
\newcommand*{\letdnsc}{\textsc{LetDn}\xspace}
\newcommand{\fix}{\keyword{fix}\xspace}
\newcommand{\up}{\sfop{up}}
\newcommand{\down}{\sfop{down}}

% Auxiliary

% https://tex.stackexchange.com/a/268475/52414
\newlength\stextwidth
\newcommand\makesamewidth[3][c]{%
  \settowidth{\stextwidth}{#2}%
  \makebox[\stextwidth][#1]{#3}%
}
\newlength\smathtextwidth
\newcommand\mathmakesamewidth[3][c]{%
  \settowidth{\smathtextwidth}{$#2$}%
  \mathmakebox[\smathtextwidth][#1]{#3}%
}
\newcommand\mathwithin[2]{%
  \mathmakesamewidth[c]{#1}{#2}
}

\definecolor{mygreen}{HTML}{6aa84f}
\newcommand{\sfop}[1]{\mathsf{#1}}
\newcommand{\uscore}[1][\mathunderscore]{\mathwithin{#1}{\mathunderscore}}
\newcommand{\expand}{\sfop{expand}}
\newcommand{\dom}[1]{\sfop{dom}\,#1}
\newcommand{\zap}{\sfop{zap}}
\newcommand{\letup}[2]{#1\!\uparrow_{#2}}
\newcommand{\letdown}[2]{#1\!\downarrow_{#2}}
\newcommand{\liftstar}[1]{[#1]^*}
\newcommand{\liftqm}[1]{[#1]^\sfop{?}}
\newcommand{\cmblet}[2]{#1\ltimes#2}

% Map Notation
\newcommand{\pfun}{\rightharpoonup}
\newcommand{\emptymap}{[]}
\newcommand{\id}[1]{#1}
\newcommand{\maplit}[3][\id]{\left[#1{#2\mapsto#3}\right]}
\newcommand{\restrict}[2]{#1\restriction_{#2}}


\title{Call Arity vs. Demand Analysis}
\author{Sebastian Graf}
\subtitle{\insertauthor}
\institute[IPD]{Lehrstuhl Programmierparadigmen, IPD Snelting}
\date{9.8.2017}
\KITtitleimage{cover.png}

\begin{document}

\begin{frame}
  \maketitle
\end{frame}

\begin{frame}{Motivation}
  \begin{itemize}
    \item Glasgow Haskell Compiler (GHC): >100k lines of code
    \item Typical challenges of big software projects
    \item Two analyses with overlapping concerns
      \begin{itemize}
        \item Demand Analyser
        \item Call Arity
      \end{itemize}
    \item Extract overlaps into combined analysis
      \begin{itemize}
        \item Both prior analyses profit from increased precision
        \item Principle of DRY
      \end{itemize}
  \end{itemize}
\end{frame}

\begin{frame}[fragile]{Cardinality Analysis}
  \begin{itemize}
    \item 
      Provides lower and upper bounds on \emph{evaluation cardinality}
      \begin{itemize} \item \MinCard and \MaxCard analyses \end{itemize}
    \item Notation: $[n,m]$, where $n,m \in \{0,1,\omega\}, n \leq m$
    \item \emph{Single-entry} thunks: $[\uscore,1]$
  \end{itemize}
  \begin{center}
    \begin{minipage}{0.5\textwidth}
      \begin{haskell}
        main = do
          let a = ... -- $[1,\omega]$
              b = ... -- $[1,1]$
              c = ... -- $[0,\omega]$
              d = ... -- $[0,1]$
              e = ... -- $[0,0]$
          print (a + if b then a * d else c * c)
      \end{haskell}
    \end{minipage}
  \end{center}
\end{frame}

\begin{frame}[fragile]{Usage Analysis}
  \begin{itemize}
    \item \MaxCard analysis, \eg computes $m$ in binding annotations $[n,m]$
    \item Related: \emph{one-shot} lambdas
      \begin{itemize}
        \item \hs{let} Floating
        \item $\eta$-Expansion
        \item Inlining
      \end{itemize}
  \end{itemize}
  \begin{center}
    \begin{minipage}{0.5\textwidth}
      \begin{haskell*}{escapeinside=||}
        main = do        
          let c = ...
              f = \|$\mathtt{x}^\omega$| |$\mathtt{y}^1$| -> c*x + y
              g = \|$\mathtt{z}^1$| -> 
                    let d = ... 
                    in \|$\mathtt{w}^1$| -> c*d + z*w
          print (f 1 2 + f 4 5 * g 3 42)
      \end{haskell*}
    \end{minipage}
  \end{center}
\end{frame}

\begin{frame}[fragile]{Demand Analyser}
  \begin{itemize}
    \item A cardinality analysis
    \item Usage analysis identifies
      \begin{itemize}
        \item Single-entry thunks
        \item One-shot lambdas
        \item Dead code
      \end{itemize}
    \item Computes how an expression uses its free variables
  \end{itemize}
\end{frame}

\begin{frame}[fragile]{Demand Analyser}
  \framesubtitle{Call Uses}
  \begin{itemize}
    \item $n \in \sMulti \Coloneqq 1 \mid \omega$
    \item $u \in \sUse \Coloneqq C^n(u) \mid U$
    \item $\sUsage \Coloneqq A \mid n*u$
    \item \hs{f 1 2} exposes \hs{f} to usage $1*C^1(C^1(U))$
    \item \hs{f 1 2 + f 4 5} exposes \hs{f} to usage $\omega*C^\omega(C^1(U))$
  \end{itemize}
  \begin{center}
    \begin{minipage}{0.5\textwidth}
      \begin{haskell*}{escapeinside=||}
        let f |$\mathtt{x}^\omega$| |$\mathtt{y}^1$| = ... -- $\omega*C^\omega(C^1(U))$
        in f 1 2 + f 4 5
      \end{haskell*}
    \end{minipage}
  \end{center}
\end{frame}

\begin{frame}[fragile]{Demand Analyser}
  \framesubtitle{Usage Signatures}
  \begin{itemize}
    \item Interprocedural data-flow
    \item Looks at the RHS of \hs{const} before the \hs{let} body!
    \item \emph{Usage type} ($\sUType$): $\lPair{\text{Free variable \sUsage}}{\text{Usage signature}}$
    \item \textsc{LetDn} rule: \emph{Unleashes} usage type of \hs{const}'s RHS at call sites
  \end{itemize}

  \begin{center}
    \begin{minipage}{0.8\textwidth}
      \begin{haskell}
        let const a b = a -- $\mathtt{const}_{rhs} :: \lPair{\emptymap}{1*U \to A \to \bullet}$
        in const 
            y             -- $1*U$
            (fac 1000)    -- $A$
      \end{haskell}
    \end{minipage}
  \end{center}
\end{frame}

\begin{frame}[fragile]{Demand Analyser}
  \framesubtitle{Thunks}
  \begin{itemize}
    \item \hs{x} is only used once, its evaluation in \hs{y}'s right-hand side is shared!
    \item 
      Two use sites 
      \begin{itemize} 
        \item \textsc{LetDn}-style leads to usage of $\omega*U$ on \hs{x}! 
      \end{itemize}
    \item<2-3> Solution: \textsc{LetUp} for thunks
      \begin{itemize} 
        \item Record usage $\omega*U$ of \hs{y} in the body of the \hs{let}
        \item Analyse \hs{y}'s bound expression once in use $U$, unleash its usage type
      \end{itemize}
  \end{itemize}
  \begin{overprint}
    \onslide<1>
    \begin{center}
      \begin{minipage}{0.6\textwidth}
        \begin{haskell*}{escapeinside=||}
          let y = 2*x -- $\mathtt{y}_{rhs} :: \lPair{\maplit{\mathtt{x}}{1*U}}{\bullet}$
          in y + y
        \end{haskell*}
      \end{minipage}
    \end{center}

    \onslide<2>
    \begin{center}
      \begin{minipage}{0.6\textwidth}
        \begin{haskell*}{escapeinside=||}
          let y = 2*x 
          in y + y    -- $:: \lPair{\maplit{\mathtt{y}}{\omega*U}}{\bullet}$
        \end{haskell*}
      \end{minipage}
    \end{center}

    \onslide<3>
    \begin{center}
      \begin{minipage}{0.6\textwidth}
        \begin{haskell*}{escapeinside=||}
          let y = 2*x -- $:: \lPair{\maplit{\mathtt{x}}{1*U}}{\bullet}$ 
          in y + y    
        \end{haskell*}
      \end{minipage}
    \end{center}
  \end{overprint}
\end{frame}

\begin{frame}[fragile]{Call Arity}
  \begin{overprint}
    \onslide<1>
    \begin{itemize}
      \item $\eta$-expansion based on usage information to guarantee safety
      \item \hs{f} is always called with two arguments
      \item<2> \hs{f}'s bound expression can be $\eta$-expanded!
    \end{itemize}

    \onslide<2>
    \begin{itemize}
      \item $\eta$-expansion based on usage information to guarantee safety
      \item \hs{f} is always called with two arguments
      \item<2> \hs{f}'s bound expression can be $\eta$-expanded!
    \end{itemize}

    \onslide<3>
    \begin{itemize}
      \item Can we $\eta$-expand \hs{f} here?
      \item \hs{f} is always called with two arguments
    \end{itemize}

    \onslide<4>
    \begin{itemize}
      \item Can we $\eta$-expand \hs{f} here?
      \item \hs{f} is always called with two arguments
      \item ... but \hs{f} is also a thunk
      \item $\eta$-expansion would duplicate the \hs{expensive} check!
    \end{itemize}

    \onslide<5>
    \begin{itemize}
      \item Can we $\eta$-expand \hs{f} here?
      \item \hs{f} is a thunk, called with two arguments
    \end{itemize}

    \onslide<6>
    \begin{itemize}
      \item Can we $\eta$-expand \hs{f} here?
      \item \hs{f} is a thunk, called with two arguments
      \item Since \hs{f} is only \emph{called once}, $\eta$-expansion is safe
    \end{itemize}
  \end{overprint}

  \begin{overprint}
    \onslide<1>
    \begin{center}
      \begin{minipage}{0.5\textwidth}
        \begin{haskell}
          let f x = 
                if expensive
                then \y -> y
                else \y -> y*x
          in f 1 2 + f 4 5
        \end{haskell}
      \end{minipage}
    \end{center}

    \onslide<2>
    \begin{center}
      \begin{minipage}{0.5\textwidth}
        \begin{haskell}
          let f x y = 
                if expensive
                then y
                else y*x
          in f 1 2 + f 4 5
        \end{haskell}
      \end{minipage}
    \end{center}

    \onslide<3>
    \begin{center}
      \begin{minipage}{0.5\textwidth}
        \begin{haskell}
          let f = 
                if expensive
                then \x y -> y
                else \x y -> y*x
          in f 1 2 + f 4 5
        \end{haskell}
      \end{minipage}
    \end{center}

    \onslide<4>
    \begin{center}
      \begin{minipage}{0.5\textwidth}
        \begin{haskell}
          let f = 
                if expensive
                then \x y -> y
                else \x y -> y*x
          in f 1 2 + f 4 5
        \end{haskell}
      \end{minipage}
    \end{center}

    \onslide<5>
    \begin{center}
      \begin{minipage}{0.5\textwidth}
        \begin{haskell}
          let f = 
                if expensive
                then \x y -> y
                else \x y -> y*x
          in f 1 2
        \end{haskell}
      \end{minipage}
    \end{center}

    \onslide<6>
    \begin{center}
      \begin{minipage}{0.5\textwidth}
        \begin{haskell}
          let f x y = 
                if expensive
                then y
                else y*x
          in f 1 2
        \end{haskell}
      \end{minipage}
    \end{center}
  \end{overprint}
\end{frame}

\begin{frame}[fragile]{Call Arity}
  \framesubtitle{Co-call Graphs}
  \begin{itemize}
    \item Called-once information is tracked by a usage analysis
    \item Like \textsc{LetUp}: Analyses \hs{let} body before RHS, bottom-up
    \item Ordinary \textsc{LetUp}-style usage analysis too imprecise
    \item \emph{Co-call graphs} track control flow information
  \end{itemize}

  \begin{columns}
    \begin{column}{0\textwidth}
      \begin{haskell}
        let x = ...  
        in let y = 2*x
           in if b
              then x
              else y
      \end{haskell}
    \end{column}
    \begin{column}{0\textwidth}
      \begin{tikzpicture}
        \node at (0,0) (0) {\hs{b}};
        \node at (1,0.3) (1) {\hs{x}};
        \node at (1,-0.3) (2) {\hs{y}};
        \draw (0) -- (1);
        \draw (0) -- (2);
      \end{tikzpicture}
    \end{column}
  \end{columns}
\end{frame}

\begin{frame}{Combined Usage Analysis}
  \begin{itemize}
    \item Idea: Extend usage analysis of the Demand Analyser
    \item Add co-call graphs to usage types for tracking single-entry thunks
    \item \textsc{LetDn}-style analysis with precise call uses for functions
  \end{itemize}
\end{frame}

\begin{frame}[fragile]{Call Arity vs. Demand Analysis}
  \begin{overprint}
    \onslide<1>
    \begin{itemize}
      \item Free variable \hs{z} is used only once
      \item Recall: Demand Analysis analyses function bindings top-down (\textsc{LetDn})
      \item \textsc{LetDn} only digests \hs{f} for manifest arity 1, can't look under lambda
      \item \hs{f} is called with 2 arguments, $C^1(C^1(U))$
    \end{itemize}

    \onslide<2>
    \begin{itemize}
      \item[\xmark] Demand Analysis: \hs{z} is annotated as used multiple times
      \item[\cmark] Call Arity (\textsc{LetUp}): Arity information flows bottom-up, \hs{z} used once
      \item Solution: Analyse bound function on demand, when we know specific call use 
    \end{itemize}
  \end{overprint}

  \begin{overprint}
    \onslide<1>
    \begin{center}
      \begin{minipage}{0.7\textwidth}
        \begin{haskell*}{escapeinside=||}
          let f x = -- $\mathtt{f}_{rhs} :: \lPair{\maplit{\mathtt{z}}{\omega*U}}{\omega*U\to \bullet}$
                if expensive
                then \y -> y*|\color{red}\texttt{z}|
                else \y -> y*x
          in f 1 2
        \end{haskell*}
      \end{minipage}
    \end{center}

    \onslide<2>
    \begin{center}
      \begin{minipage}{0.7\textwidth}
        \begin{haskell*}{escapeinside=||}
          let f x = -- $\mathtt{f}_{rhs} :: \lPair{\maplit{\mathtt{z}}{1*U}}{1*U\to 1*U \to \bullet}$
                if expensive
                then \y -> y*|\color{red}\texttt{z}|
                else \y -> y*x
          in f 1 2
        \end{haskell*}
      \end{minipage}
    \end{center}
  \end{overprint}
\end{frame}

\begin{frame}[fragile]{Combined Usage Analysis}
  \begin{itemize}
    \item  \textsc{LetDn}-style analysis with precise call uses for functions
      \begin{itemize} \item Analyse bound function on demand, when we know specific call use \end{itemize}
    \item Recursion leads to termination problems
    \item Rediscovered fixed-point iteration, detached from the syntax tree
    \item Led to graph-based abstraction solved by worklist algorithm
  \end{itemize}
  \begin{center}
    \begin{minipage}{0.5\textwidth}
      \begin{haskell}
        let fac n = 
              if n == 0
              then 1
              else n * fac (n-1)
        in fac 12
      \end{haskell}
    \end{minipage}
  \end{center}
\end{frame}

\begin{frame}{Transfer Function}
  \begin{itemize}
    \item Usage type depends on how the expression is used (\eg arity)
    \item Specification denotes expressions by \newdef{usage transformers}
    \item $\sUTrans = \sUse \to \sUType$
    \item $\transfer{\uscore}{\uscore[\rho]} : \sExp \to \sTransEnv \to \sUTrans$
  \end{itemize}
  \begin{alignat*}{2}
    \transfer{\sLet{\bind}{e}}{\rho}\,u &{}={}&& (\cmblet{\theta}{\maplit{x_1}{\theta_1}}) \setminus_{x_1} \\
    \text{where} \\
     \tau_1 &{}={}&& \transfer{e_1}{\rho} \\
     \rho' &{}={}&& \maplit{x_1}{\letdown{\tau_1}{e_1}}\rho \\ 
     \theta &{}={}&& \transfer{e}{\rho'}\,u \\ 
     \theta_1 &{}={}&& \liftqm{\letup{\tau_1}{e_1}}{\theta(x_1)}
  \end{alignat*}
\end{frame}

\begin{frame}[fragile]{Recovering Call Arity}
  \begin{itemize}
    \item Observation: (Safe) $\eta$-expansion does not change how an expression is used
    \item Perform $\eta$-Expansion based on Usage results $n*u$
  \end{itemize}
  \begin{overprint}
    \onslide<1>
    \begin{center}
      \begin{minipage}[t][10cm][t]{0.5\textwidth}
        \begin{haskell}
          let f x =   -- $\omega*C^\omega(C^{\color{red} 1}(U))$
                if b
                then \y -> y
                else \y -> y*x
          in f 1 2 + f 4 5
        \end{haskell}
      \end{minipage}
    \end{center}

    \onslide<2>
    \begin{center}
      \begin{minipage}{0.5\textwidth}
        \begin{haskell*}{escapeinside=||}
          let f x |\color{mygreen}\texttt{y}| = -- $\omega*C^\omega(C^1(U))$
                if b
                then y
                else y*x
          in f 1 2 + f 4 5
        \end{haskell*}
      \end{minipage}
    \end{center}

    \onslide<3>
    \begin{center}
      \begin{minipage}{0.5\textwidth}
        \begin{haskell}
          let t =     -- ${\color{red} \omega}*C^\omega(C^1(U))$
                if b
                then \x y -> y
                else \x y -> y*x 
          in t 1 2 + t 4 5
        \end{haskell}
      \end{minipage}
    \end{center}

    \onslide<4>
    \begin{center}
      \begin{minipage}{0.5\textwidth}
        \begin{haskell}
          let t =     -- ${\color{red} 1}*C^1(C^1(U))$
                if b
                then \x y -> y
                else \x y -> y*x 
          in t 1 2
        \end{haskell}
      \end{minipage}
    \end{center}

    \onslide<5>
    \begin{center}
      \begin{minipage}{0.5\textwidth}
        \begin{haskell}
          let t =     -- $1*C^{\color{red} 1}(C^1(U))$
                if b
                then \x y -> y
                else \x y -> y*x 
          in t 1 2
        \end{haskell}
      \end{minipage}
    \end{center}

    \onslide<6>
    \begin{center}
      \begin{minipage}{0.5\textwidth}
        \begin{haskell*}{escapeinside=||}
          let t |\color{mygreen}\texttt{x}| =   -- $1*C^1(C^{\color{red} 1}(U))$
                if b
                then \y -> y
                else \y -> y*x 
          in t 1 2
        \end{haskell*}
      \end{minipage}
    \end{center}

    \onslide<7>
    \begin{center}
      \begin{minipage}{0.5\textwidth}
        \begin{haskell*}{escapeinside=||}
          let t x |\color{mygreen}\texttt{y}| = -- $1*C^1(C^1(U))$
                if b
                then y
                else y*x
          in t 1 2
        \end{haskell*}
      \end{minipage}
    \end{center}
  \end{overprint}
\end{frame}

\begin{frame}[fragile]{Implementation}
  \begin{itemize}
    \item Central part of GHC, changes with many repercussions
    \item Co-call graphs: Efficient representation of sparse and dense cases
    \item Solving the data-flow problem through graph-based fixed-point iteration
  \end{itemize}
\end{frame}

\begin{frame}{Evaluation}
  \begin{itemize}
    \item Benchmarked two variants, baseline GHC 8.2.1
      \begin{itemize}
        \item Full analysis, with co-call graphs \varfull
        \item Without co-call graphs, but with use-sensitive \textsc{LetDn} \varedges
      \end{itemize}
    \item Measured allocations and executed instructions
      \begin{itemize}
        \item Artifact performance on nofib suite
        \item Compiler performance compiling nofib and compiling itself
      \end{itemize}
  \end{itemize}
\end{frame}

\begin{frame}{Evaluation}
  \framesubtitle{Artifact Performance}
  \begin{columns}
    \begin{column}{0.4\textwidth}
      \begin{itemize}
        \item No regressions relative to the baseline (technically)
        \item Co-call graphs didn't influence allocations noticeably
      \end{itemize}
    \end{column}
    \begin{column}{0.6\textwidth}
      \begin{table}
        \centering
        \begin{tabular}{lrr}
          \toprule
                  & \multicolumn{2}{c}{Bytes allocated} \\
                    \cmidrule(lr){2-3}
          Program & \multicolumn{1}{c}{\varfull} & \multicolumn{1}{c}{\varedges} \\
          \midrule
          \progname{fft2} & -0.9\% & -0.9\% & -0.9\%\\
\progname{fluid} & -1.5\% & -1.5\% & -1.5\%\\
\progname{infer} & +0.0\% & +0.0\% & +0.0\%\\
\progname{n-body} & -0.2\% & -0.2\% & -0.2\%\\
\progname{reptile} & -0.1\% & -0.1\% & -0.1\%\\
\progname{rfib} & -0.1\% & -0.1\% & -0.1\%\\
\progname{spectral-norm} & -0.1\% & -0.1\% & -0.1\%\\
\progname{tak} & -0.4\% & -0.4\% & -0.4\%\\
\andmore{95} & & & & & & \\
\midrule
\totalname{Min} & -1.5\% & -1.5\% & -1.5\%\\
\totalname{Max} & +0.0\% & +0.0\% & +0.0\%\\
\totalname{Geometric Mean} & -0.0\% & -0.0\% & -0.0\%\\



          \bottomrule
        \end{tabular}
      \end{table}
    \end{column}
  \end{columns}
\end{frame}

\begin{frame}{Evaluation}
  \framesubtitle{Compiler Performance}
  \begin{itemize}
    \item Compiler performance got worse
      \begin{itemize}
        \item One run of Call Arity replaced by three runs of our usage analysis
      \end{itemize}
    \item Space usage of variant \varfull goes through the roof for some inputs
  \end{itemize}
  \begin{table}
    \centering
    \begin{tabular}{lrrrr}
      \toprule
              & \multicolumn{2}{c}{Bytes allocated} & \multicolumn{2}{c}{Instructions executed} \\
                \cmidrule(lr){2-3} \cmidrule(lr){4-5}
      Program & \multicolumn{1}{c}{\varfull} & \multicolumn{1}{c}{\varedges} & \multicolumn{1}{c}{\varfull} & \multicolumn{1}{c}{\varedges} \\
      \midrule
      \totalname{nofib} & +29.8\% & +26.9\% & +11.0\% & +27.5\% & +24.5\% & +11.3\%\\

      \totalname{ghc} & +2.4\% & +2.3\% & +1.5\% & +2.6\% & +2.0\% & +2.3\% \\

      \bottomrule
    \end{tabular}
  \end{table}
\end{frame}

\begin{frame}{Related Work}
  \begin{itemize}
    \item Abstract Interpretation
      \begin{itemize}
        \item Call Arity \parencite{callarity}
        \item Demand Analysis \parencite{card}
        \item 80's and 90's: Projection-based strictness analysis \parencite{wadlerproj,projimpl,projs}
      \end{itemize}
    \item Type Systems
      \begin{itemize}
        \item Uniqueness types 
        \item 90's until today \parencite{type,polytype,warnsbrough,sharing}
        \item Recent thesis by \textcite{verstoepthesis}, planned integration into UHC \parencite{verstoep}
        \item Polymorphism, subtyping, data type annotations problematic for performance \parencite{card}
      \end{itemize}
  \end{itemize}
\end{frame}

\begin{frame}{Conclusion}
  \begin{itemize}
    \item Usage analysis generalising Call Arity and its counterpart in the Demand Analyser
      \begin{itemize}
        \item No regressions, but no real improvement either
      \end{itemize}
    \item 
      \texttt{\$ git diff -{}-shortstat ghc-8.2.1-release cocall-full}\\
      \texttt{52 files changed, 3012 insertions(+), 1053 deletions(-)}
    \item Future work
      \begin{itemize}
        \item `Novel' approach to fixed-point iteration, detached from syntax tree
        \item Data structure for (monotone) maps, indexed by partial order
        \item Long term: Integration into GHC
      \end{itemize}
  \end{itemize}
\end{frame}

\begin{frame}[shrink=15]{Bibliography} 
  \printbibliography
\end{frame}

\setbeamertemplate{section page}{%
  \begin{centering}
  \begin{beamercolorbox}[sep=12pt,center]{part title}
  \usebeamerfont{section title}\insertsection\par
  \end{beamercolorbox}
  \end{centering}
}
\section{Backup}
\frame{\sectionpage}

\begin{frame}[fragile]{Call-by-value}
  \begin{itemize}
    \item 
      Avoids operational complexity of laziness, enables unboxing
      \begin{itemize} \item Exploits \emph{strictness} \end{itemize}
      \begin{itemize} \item `Evaluated at \emph{least} once' (\MinCard) \end{itemize}
  \end{itemize}
  \begin{center}
    \begin{minipage}{0.5\textwidth}
      \begin{haskell}
        main = do
          let !a    = ...                 -- $[\mathbf{1},\omega]$
              !b    = ...                 -- $[\mathbf{1},1]$
               c    = ...                 -- $[0,\omega]$
               d    = ...                 -- $[0,1]$
               e    = ...                 -- $[0,0]$
          print (a + if b then a * d else c * c)
      \end{haskell}
    \end{minipage}
  \end{center}
\end{frame}

\begin{frame}[fragile]{Call-by-name}
  \begin{itemize}
    \item 
      Omits updates of \emph{single-entry} thunks, mimicing regular function calls
      \begin{itemize} \item Exploits lack of \emph{sharing} \end{itemize}
      \begin{itemize} \item `Evaluated at \emph{most} once' (\MaxCard) \end{itemize}
  \end{itemize}
  \begin{center}
    \begin{minipage}{0.5\textwidth}
      \begin{haskell}
        main = do
          let  a    = ...                 -- $[1,\omega]$
               b () = ...                 -- $[1,\mathbf{1}]$
               c    = ...                 -- $[0,\omega]$
               d () = ...                 -- $[0,\mathbf{1}]$
               e () = ...                 -- $[0,\mathbf{0}]$
          print (a + if b () then a * d () else c * c)
      \end{haskell}
    \end{minipage}
  \end{center}
\end{frame}

\begin{frame}[fragile]{Dead code}
  \begin{itemize}
    \item Can replace dead bindings with small traps
    \begin{itemize} \item Exploits \emph{absence} \end{itemize}
    \begin{itemize} \item `Never evaluated' (\MaxCard) \end{itemize}
  \end{itemize}
  \begin{center}
    \begin{minipage}{0.5\textwidth}
      \begin{haskell}
        main = do
          let  a    = ...                 -- $[1,\omega]$
               b    = ...                 -- $[1,1]$
               c    = ...                 -- $[0,\omega]$
               d    = ...                 -- $[0,1]$
               e    = error "e is absent" -- $[0,\mathbf{0}]$
          print (a + if b then a * d else c * c)
      \end{haskell}
    \end{minipage}
  \end{center}
\end{frame}

% Proably backup, also useful to explain inlining
\begin{frame}[fragile]{Floating}
  \begin{itemize}
    \item Floating bindings in and out of expressions
    \item Floating into a lambda possibly duplicates work!
    \item Not so with one-shot lambdas
  \end{itemize}
  \begin{center}
    \begin{minipage}{0.5\textwidth}
      \begin{overprint}
        \onslide<1>
        \begin{haskell*}{escapeinside=||}
          main = do        
            let c = ...
                f = \|$\mathtt{x}^\omega$| |$\mathtt{y}^1$| -> c*x + y
                g = \|$\mathtt{z}^1$| -> 
                      let d = ... 
                      in \|$\mathtt{w}^1$| -> c*d + z*w
            print (f 1 2 + f 4 5 * g 3 42)
        \end{haskell*}
        \onslide<2>
        \begin{haskell*}{escapeinside=||}
          main = do        
            let |\color{red}\texttt{c}| = ...
                f = \|$\mathtt{x}^{\color{red} \omega}$| |$\mathtt{y}^1$| -> |\color{red}\texttt{c}|*x + y
                g = \|$\mathtt{z}^1$| -> 
                      let d = ... 
                      in \|$\mathtt{w}^1$| -> |\color{red}\texttt{c}|*d + z*w
            print (f 1 2 + f 4 5 * g 3 42)
        \end{haskell*}
        \onslide<3>
        \begin{haskell*}{escapeinside=||}
          main = do        
            let c = ...
                f = \|$\mathtt{x}^\omega$| |$\mathtt{y}^1$| -> c*x + y
                g = \|$\mathtt{z}^1$| -> 
                      let |\color{red}\texttt{d}| = ... 
                      in \|$\mathtt{w}^1$| -> c*|\color{red}\texttt{d}| + z*w
            print (f 1 2 + f 4 5 * g 3 42)
        \end{haskell*}
        \onslide<4>
        \begin{haskell*}{escapeinside=||}
          main = do        
            let c = ...
                f = \|$\mathtt{x}^\omega$| |$\mathtt{y}^1$| -> c*x + y
                g = \|$\mathtt{z}^1$| |$\mathtt{w}^1$| -> 
                      let |\color{red}\texttt{d}| = ... 
                      in c*|\color{red}\texttt{d}| + z*w
            print (f 1 2 + f 4 5 * g 3 42)
        \end{haskell*}
      \end{overprint}
    \end{minipage}
  \end{center}
\end{frame}

\begin{frame}[fragile]{$\eta$-Expansion}
  \begin{itemize}
    \item Floats lambdas out, possibly duplicating work
    \item Not so with one-shot lambdas
  \end{itemize}
  \begin{center}
    \begin{minipage}{0.5\textwidth}
      \begin{overprint}
        \onslide<1>
        \begin{haskell*}{escapeinside=||}
          main = do        
            let g = \|$\mathtt{z}^1$| -> 
                      let d = ... 
                      in \|$\mathtt{w}^{\color{red} 1}$| -> d + z*w
            print (g 3 42)
        \end{haskell*}
        \onslide<2>
        \begin{haskell*}{escapeinside=||}
          main = do        
            let g = \|$\mathtt{z}^1$| |\color{red}$\mathtt{w}^1$| -> 
                      let d = ... 
                      in d + z*w
            print (g 3 42)
        \end{haskell*}
      \end{overprint}
    \end{minipage}
  \end{center}
\end{frame}

\begin{frame}[fragile]{Call Arity}
  \framesubtitle{Co-call Graphs}
  \begin{overprint}
    \onslide<1>
    \begin{itemize}
      \item Called-once information is tracked by a usage analysis
      \item Analyses \hs{let} body before bound expression, bottom-up, like \textsc{LetUp}
      \item Ordinary \textsc{LetUp}-style usage analysis too imprecise
        \begin{itemize}
          \item Forgets that \hs{x} and \hs{y} were evaluated on different code paths
          \item Analysis reports \hs{x} to be used multiple times
        \end{itemize}
    \end{itemize}

    \onslide<2>
    \begin{itemize}
      \item \emph{Co-call graphs} track control flow information
        \begin{itemize}
          \item Edges connect variables possible evaluated on the same code path
          \item Absence of an edge proves mutual exclusive calls
          \item Usage information in self-edges
        \end{itemize}
      \item Lack of edge between \hs{x} and \hs{y} avoids self-edge on \hs{x}
    \end{itemize}

    \onslide<3>
    \begin{itemize}
      \item \emph{Co-call graphs} track control flow information
        \begin{itemize}
          \item Edges connect variables possible evaluated on the same code path
          \item Absence of an edge proves mutual exclusive calls
          \item Usage information in self-edges
        \end{itemize}
      \item Lack of edge between \hs{x} and \hs{y} avoids self-edge on \hs{x}
    \end{itemize}

    \onslide<4>
    \begin{itemize}
      \item \emph{Co-call graphs} track control flow information
        \begin{itemize}
          \item Edges connect variables possible evaluated on the same code path
          \item Absence of an edge proves mutual exclusive calls
          \item Usage information in self-edges
        \end{itemize}
      \item Lack of edge between \hs{x} and \hs{y} avoids self-edge on \hs{x}
    \end{itemize}
  \end{overprint}

  \begin{columns}
    \begin{column}{0\textwidth}
      \begin{haskell}
        let x = ...  
        in let y = 2*x
           in if b
              then x
              else y
      \end{haskell}
    \end{column}
    \begin{column}{0\textwidth}
      \begin{overprint}
        \onslide<2>
        \begin{tikzpicture}
          \node at (0,0) (0) {\hs{b}};
          \node at (1,0.3) (1) {\hs{x}};
          \node at (1,-0.3) (2) {\hs{y}};
          \draw (0) -- (1);
          \draw (0) -- (2);
        \end{tikzpicture}

        \onslide<3>
        \begin{tikzpicture}
          \node at (0,0) (0) {\hs{b}};
          \node at (1,0.3) (1) {\hs{x}};
          \node at (1,-0.3) (2) {\color{red}\texttt{x}};
          \draw (0) -- (1);
          \draw (0) -- (2);
        \end{tikzpicture}

        \onslide<4>
        \begin{tikzpicture}
          \node at (0,0) (0) {\hs{b}};
          \node at (1,0) (1) {\hs{x}};
          \draw (0) -- (1);
        \end{tikzpicture}
      \end{overprint}
    \end{column}
  \end{columns}
\end{frame}

\begin{frame}[fragile]{\letupsc and \hs{seq}}
  $\mathtt{seq}_{rhs} :: \lPair{\emptymap}{1*HU \to 1*U \to \bullet}$
  \begin{center}
    \begin{minipage}{0.4\textwidth}
      \begin{haskell}
        let t = -- $\omega*C^1(U)$
              if expensive
              then \x -> x
              else \x -> x*2
        in seq t (t 15)
      \end{haskell}
    \end{minipage}
  \end{center}
\end{frame}

\begin{frame}[fragile]{Fusion}
  \begin{center}
    \begin{minipage}{0.6\textwidth}
      \begin{haskell}
        let go x =
             let r = 
                  if x == 2016
                  then id 
                  else go (x + 1)
             in if f x 
                then \a -> r (a + x)
                else r
        in go 42 0
      \end{haskell}
    \end{minipage}
  \end{center}
\end{frame}

\begin{frame}[fragile]{Strict Improvement}
  \begin{center}
    \begin{minipage}{0.7\textwidth}
      \begin{overprint}
        \onslide<1>
        \begin{haskell}
          let f x =
                  if expensive 
                  then \y -> y 
                  else \y -> y*x
          in let g f = f
             in g f 1 2
        \end{haskell}

        \onslide<2>
        \begin{haskell}
          let f x y =
                  if expensive 
                  then y 
                  else y*x
          in let g f x y = f x y
             in g f 1 2
        \end{haskell}
      \end{overprint}
    \end{minipage}
  \end{center}
\end{frame}

\begin{frame}[fragile]{Graph-based Fixed-point iteration}
  \begin{columns}
    \begin{column}{0.4\textwidth}
      \begin{center}
        \begin{overprint}
          \onslide<1>
          \begin{haskell}
            module Main (main) where

            fac n = 
              if n == 0 
              then 1 
              else n * fac (n - 1)

            main = print (fac 12)
          \end{haskell}
          \onslide<2>
          \begin{haskell}
            let fac n =
                  if n == 0
                  then 1
                  else n * fac (n - 1)
            in let main = print (fac 12)
               in main
          \end{haskell}
        \end{overprint}
      \end{center}
    \end{column}
    \begin{column}{0.6\textwidth}
      \begin{figure}
        \centering
        \begin{tikzpicture}[shorten >=1pt,x=10em,y=7em,auto,scale=0.5, every node/.style={scale=0.5}]
            \tikzset{every loop/.style={looseness=4}}
            \tikzstyle{n}=[draw,shape=circle,minimum size=5em,inner sep=0pt]
            \tikzstyle{root}=[n,fill=myred]
            \tikzstyle{up}=[n,fill=myblue]
            \tikzstyle{down}=[n,fill=mygreen]
            \node (root) at (0,2) [root] {$(\hs{root}, U)$};
            \node (let1) at (1,2) [up] {$(\keyword{let}_1, U)$};
            \node (let2) at (2,2) [up] {$(\keyword{let}_2, U)$};
            \node (main) at (2,1) [down] {$(\hs{main}, U)$};
            \node (fac1) at (1.5,0) [down] {$(\hs{fac}, C^1(U))$};
            \node (facn) at (1,1) [down] {$(\hs{fac}, U)$};
            \path[->,line width=0.5pt] 
              (root) edge node {} (let1)
              (let1) edge [loop above] node {} ()
                     edge node {} (let2)
                     edge node {} (facn)
              (let2) edge node {} (main)
              (main) edge node {} (fac1)
              (facn) edge node {} (fac1)
              (fac1) edge [loop below] node {} ();
        \end{tikzpicture}
      \end{figure}
    \end{column}
  \end{columns}
\end{frame}

\begin{frame}[fragile]{\letupsc and \hs{seq}}
  \begin{center}
    \begin{minipage}{0.4\textwidth}
      \begin{haskell}
        let f = 
              if expensive
              then id
              else (*2)
        in let x = f 0
           in if b
              then f `seq` x
              else f 0
      \end{haskell}
    \end{minipage}
  \end{center}
\end{frame}

\begin{frame}[fragile]{Recovering Call Arity}
  \begin{itemize}
    \item Observation: (Safe) $\eta$-expansion does not change how an expression is used
    \item Perform $\eta$-Expansion based on Usage results $n*u$
  \end{itemize}
  \[
    \begin{array}{lcl}
      \expand_{\sUsage}                              &\,\colon& \sUsage \to \mathbb{N} \to \mathbb{N} \\
      \expand_{\sUsage}\,A\,\alpha                   &    =   & \alpha \\
      \expand_{\sUsage}\,(\omega*\uscore[u])\,0      &    =   & 0 \\
      \expand_{\sUsage}\,(\uscore[\omega]*u)\,\alpha &    =   & \expand_{\sUse}\,u\,\alpha \\
      \\
      \expand_{\sUse}                                &\,\colon& \sUse \to \mathbb{N} \to \mathbb{N} \\
      \expand_{\sUse}\,C^\omega(u)\,0                &    =   & 0 \\
      \expand_{\sUse}\,C^{\uscore}(u)\,\alpha        &    =   & 1 + \expand_{\sUsage}\,u\,(\max\,0\,(\alpha - 1)) \\
      \expand_{\sUse}\,\uscore\,\alpha               &    =   & \alpha \\
    \end{array}
  \]
\end{frame}

\begin{frame}{Lambda Abstraction, Application, Variables}
  \[
  \transfer{\sLam{x}{e}}{\rho}\,u =
    \begin{cases}
      \lTriple{\emptyset}{\emptymap}{\bot}, & \text{when } u = HU \\
      n*\lTriple{\gamma\setminus_x}{\varphi\setminus_x}{\theta(x) \to \sigma}, & \text{where } 
        \begin{array}{l}
          u \lleq C^n(u_b), \\
          \lTriple{\gamma}{\varphi}{\sigma} = \theta = \transfer{e}{\rho}\,u_b
        \end{array}
    \end{cases}
  \]

\begin{alignat*}{2}
  \transfer{\sApp{e}{x}}{\rho}\,u &=&& \lTriple{\gamma_e}{\varphi_e}{\sigma} \both \liftstar{\transfer{x}{\rho}}\,u^* \\
  &\rlap{ where $\lTriple{\gamma_e}{\varphi_e}{u^* \to \sigma} = \transfer{e}{\rho}\,C^1(u)$} &&\\
\end{alignat*}
\begin{alignat*}{2}
&\liftstar{\uscore[\tau]}~A &{}={}& \lTriple{\emptyset}{\emptymap}{\bot} \\
&\liftstar{\tau}~(n*u)      &{}={}& n*(\tau~u)
\end{alignat*}

  \[
  \transfer{x}{\rho} =
    \begin{cases}
      \rho(x) \both \tau^1_x, & \text{when } x\in \dom{\rho} \\
      \tau^1_x, & \text{otherwise}
    \end{cases}
  \]
  \[
  \tau^1_x~u = \lTriple{\emptyset}{\maplit{x}{u}}{\top}
  \]
\end{frame}

\begin{frame}{Recursive Let}
  \begin{alignat*}{2}
    \transfer{\sLet{\binds}{e}}{\rho}\,u &{}={}&& (\fix\,\up) \setminus_{\overline{x_i}} \\
    \text{where} \\
     \up\,\theta &{}={}&& \cmblet{(\transfer{e}{\rho'}\,u \llub \theta)}{\maplit[\overline]{x_i}{\theta_i}} \\
     \overline{\theta_i} &{}={}&& \overline{\liftqm{\letup{\transfer{e_i}{\rho'}}{e_i}}{(\theta(x_i) \both \theta(x_i))}} \\
     \rho' &{}={}&& \fix\,\down \\
     \down\,\rho' &{}={}&& \maplit[\overline]{x_i}{\letdown{\transfer{e_i}{\rho \llub \rho'}}{e_i}}\rho \\ 
  \end{alignat*}
  \begin{alignat*}{2}
  &\liftqm{\uscore[\tau]}~A     &{}={}& \lTriple{\emptyset}{\emptymap}{\bot} \\
  &\liftqm{\tau}~(\uscore[n]*u) &{}={}& \tau~u
  \end{alignat*}
\end{frame}

\begin{frame}{\letupsc/\letdnsc}
\begin{alignat*}{1}
&\zap~{}\colon\sUType \to \sUType \\
&\zap~\lTriple{\uscore}{\uscore}{\sigma} = \lTriple{\emptyset}{\emptymap}{\sigma}
\end{alignat*}
\begin{alignat*}{2}
  &\letup{\uscore[\tau]}{\uscore[e]} &\mathmakesamewidth[c]{{}=}{\colon} &\sUTrans \to \sExp \to \sUTrans \\
  &\letup{\tau}{e} &{}={}&
  \begin{cases}
    \zap\circ\tau, & \text{when } e \text{ is in WHNF} \\
    \tau, & \text{otherwise}
  \end{cases} \\
\\
  &\letdown{\uscore[\tau]}{\uscore[e]} &\mathmakesamewidth[c]{{}=}{\colon} &\sUTrans \to \sExp \to \sUTrans \\
  &\letdown{\tau}{e} &{}={}&
  \begin{cases}
    \zap, & \text{when } e \text{ is in WHNF} \\
    \zap\circ\tau, & \text{otherwise}
  \end{cases}
\end{alignat*}
\end{frame}


\begin{frame}{Graph Substitution}
  \begin{alignat*}{2}
    \cmblet{\uscore[\theta]}{\mathmakesamewidth[l]{\maplit{x_i}{\lTriple{\gamma_i}{\varphi_i}{\uscore}}\mu}{\uscore[\emptymap]}} &\mathmakesamewidth[c]{~~=}\colon &&\sUType \to (\sVar \pfun \sUType) \to \sUType \\
    \cmblet{\theta}{\mathmakesamewidth[l]{\maplit{x_i}{\lTriple{\gamma_i}{\varphi_i}{\uscore}}\mu}{\emptymap}} &{}={}&& \theta \\
    \cmblet{\theta}{\maplit{x_i}{\lTriple{\gamma_i}{\varphi_i}{\uscore}}\mu} &{}={}&& \lTriple{\gamma}{\varphi}{\sigma} \\
    \text{where} \\ 
      \lTriple{\gamma'}{\varphi'}{\sigma} &{}={}&& \cmblet{\theta}{\mu} \\ 
      N &{}={}&& \neighbors{x_i}{\gamma'} \\
      \gamma &{}={}&& \gamma' \llub \gamma_i \llub (N \times \dom{\varphi_i}) \\
      \varphi &{}={}&& \varphi' \llub \varphi_i \llub (\restrict{\varphi'}{N}\both \varphi_i)
  \end{alignat*}
\end{frame}

\begin{frame}[fragile]{\texttt{transferApp}}
  \begin{center}
    \begin{minipage}{0.8\textwidth}
      \begin{haskell}
        transferApp
          :: CoreExpr
          -> (Use -> TransferFunction AnalResult)
          -> (Use -> TransferFunction AnalResult)
          -> Use -> TransferFunction AnalResult
        transferApp a transfer_f transfer_a result_use = do
          (ut_f, f') <- transfer_f (mkCallUse Once result_use)
          -- peel off one argument from the usage type
          let (arg_usage, ut_f') = peelArgUsage ut_f
          (ut_a, a') <- unleashUsage a transfer_a arg_usage
          return (ut_f' `bothUsageType` ut_a, App f' a')
      \end{haskell}
    \end{minipage}
  \end{center}
\end{frame}

\begin{frame}{Evaluation - Nofib Instr}
  \begin{columns}
    \begin{column}{0.4\textwidth}
      \begin{itemize}
        \item Wild outliers, unclear why
        \item Plays out in our favour
        \item Co-call graphs make a slight difference
      \end{itemize}
    \end{column}
    \begin{column}{0.6\textwidth}
      \begin{table}
        \centering
        \begin{tabular}{lrr}
          \toprule
                  & \multicolumn{2}{c}{Instructions executed} \\
                    \cmidrule(lr){2-3}
          Program & \multicolumn{1}{c}{\varfull} & \multicolumn{1}{c}{\varedges} \\
          \midrule
          \progname{fannkuch-redux} & +11.4\% & +11.4\%\\
\progname{gen\_regexps} & -28.1\% & -28.1\%\\
\progname{maillist} & -0.2\% & +3.4\%\\
\progname{puzzle} & -15.4\% & -15.4\%\\
\progname{sphere} & -4.5\% & -4.4\%\\
\andmore{98} & & \\
\midrule
\totalname{Min} & -28.1\% & -28.1\%\\
\totalname{Max} & +11.4\% & +11.4\%\\
\totalname{Geometric Mean} & -0.6\% & -0.6\%\\



          \bottomrule
        \end{tabular}
      \end{table}
    \end{column}
  \end{columns}
\end{frame}

\end{document}
