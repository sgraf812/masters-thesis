\chapter{Evaluation}\label{sec:eval}

With implementation considerations sorted out in \cref{sec:impl}, we turn to assessing the analysis by means of the performance of the generated code in \cref{sec:bench}.
We also discuss compiler performance in \cref{sec:compperf} and see ways to trade artifact performance for faster analysis time in \cref{sec:tradeoff}.

\section{Benchmarks}\label{sec:bench}

Enough with the mathy chit chat, give me numbers!

In this section we will look at how the analysis influences benchmark results in GHC's nofib \parencite{nofib} benchmark suite.

A GHC 8.2.1 release candidate, revision \texttt{83a6dd8}, serves as a baseline. We compare to three variants of our fork\footnote{Available at https://github.com/sgraf812/ghc under the branches \texttt{cocall-full}, \texttt{cocall-approximate-calls} and \texttt{cocall-complete-graphs}.}:

\begin{enumerate}
  \item[\varfull] 
    The most precise form of our analysis. 
    Replaces Call Arity entirely and is run two additional times, immediately after the Demand Analyser. 
    The usage information the Demand Analyser produces is only accessed to check for regressions.
  \item[\varcalls]
    This variant discards any product use at call sites. 
    E.g., $C^1(U(1*U,A))$ will be truncated to $C^1(U)$. 
    This is so that less nodes in the data-flow framework are accessed, to save space and time.
    Results should still show an improvement compared to the baseline.
  \item[\varedges]
    In addition to the changes of \varcalls, replace the co-call graph by a structure which only tracks self-edges and conservatively assumes the existence of all other co-call edges.
    The intention is to measure if the worst case quadratic complexity induced by co-call graphs matters in practice.
    Note that regressions in some cases are expected, as the baseline employs Call Arity with accurate modelling of edges in co-call graphs at least between `interesting' variables \parencite[section~3.4.1]{callarity}.
\end{enumerate}

\Cref{tbl:nofib} shows the results of running nofib.
The total impact of our rather complex analysis is rather meager, with only a reduction of 0.1\% in the geometric mean over all benchmarks.
But recalling the goal of this thesis, namely unifying both Call Arity \parencite{callarity} and Cardinality Analysis \parencite{card} into a single analysis, this is good news and asserts our claims in \cref{sec:generalise}!\smallskip

All variants exhibit the major positive outliers in \texttt{fft2}, \texttt{fluid} and \texttt{queens}.
To understand what happens for \texttt{fft2}, consider the following program:
\begin{haskellcode}
  module Main (main) where

  p x = ...

  c x y = 
    if p x 
    then \z -> y 
    else \z -> c y z
  
  f x = 
    if p x 
    then c x (f (x - 1))
    else \z -> x

  main = 
    print (f 1 2)
\end{haskellcode}

Our analysis finds out that \hsinl{f} can be $\eta$-expanded to arity 2.
Neither Call Arity nor Cardinality Analysis recognises this.

Call Arity doesn't have any mechanism similar to usage signatures, so it assumes a conservative call arity of 1 for \hsinl{f} because of the call \hsinl{c x (f (x - 1))}.

Cardinality Analysis, on the other hand, has available only the usage signature for manifest arity 2, which is too conservative and unleashes a usage of $1*U$ on \hsinl{f (x - 1)} instead of $1*C^1(U)$.

Our analysis infers from the call to \hsinl{c} with incoming arity 3 a call use of $C^1(U)$ for \hsinl{f (x - 1)}, exposing \hsinl{f} to a total usage of $C^\omega(C^1(U))$, which we can leverage through arity expansion to arity 2.
Something similar happens in \texttt{fft2} within the function \hsinl{dfth}, enabling additional inlining and other transformations. \smallskip

As expected, variant \varfull does not cause regressions in allocations, except for \texttt{infer}, where a partial application (which is regarded cheap to duplicate by GHC) is $\eta$-expanded and inlined subsequently.
\tod{Discuss \varcalls when the results are there.}
Even the most approximate variant evaluated, \varedges, still yields a negligible improvement overall, but gets rid of the potential quadratic blowup of co-call graphs.

\begin{figure}[hb]
\begin{center}
  \begin{tabular}{lrrr}
    \toprule
            & \multicolumn{3}{c}{Bytes allocated} \\
              \cmidrule(lr){2-4}
    Program & \multicolumn{1}{c}{\varfull} & \multicolumn{1}{c}{\varcalls} & \multicolumn{1}{c}{\varedges} \\
    \midrule
    ansi & -0.5\% & +0.3\% & +0.1\%\\
    awards & -0.2\% & +0.1\% & +0.0\%\\
    expert & -0.2\% & +0.2\% & +0.1\%\\
    fft2 & -2.2\% & -2.2\% & -2.2\%\\
    fluid & -1.5\% & -1.5\% & -1.5\%\\
    grep & -0.6\% & +0.5\% & +0.2\%\\
    infer & +0.0\% & +0.0\% & +0.0\%\\
    n-body & -0.5\% & +0.0\% & -0.1\%\\
    queens & -0.7\% & -0.7\% & -0.7\%\\
    rfib & -0.5\% & +0.3\% & +0.1\%\\
    scc & -0.3\% &  0.0\% & -0.3\%\\
    spectral-norm & -0.3\% & +0.1\% & -0.0\%\\
    tak & -0.7\% & +0.1\% & -0.1\%\\
    x2n1 & -0.2\% & +0.1\% & +0.0\%\\
    \emph{... and 90 more} \\
    \midrule
    Min & -2.2\% & -2.2\% & -2.2\%\\
    Max & +0.0\% & +0.5\% & +0.2\%\\
    Geometric Mean & -0.1\% & -0.0\% & -0.0\%\\
    \bottomrule
  \end{tabular}
\end{center}
\caption{
  Benchmark results of running nofib where the savings in allocations were greater than 0.1\% or where allocations got worse.
  A GHC 8.2.1 release candidate on which all work is based was used as a baseline.
  Variants \varfull to \varedges are increasingly approximate, but theoretically speaking, only \varedges may possibly yield worse results than the baseline's combination of Call Arity and Demand Analysis.
}
\label{tbl:nofib}
\end{figure}

