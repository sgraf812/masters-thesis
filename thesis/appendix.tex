\chapter{Sonstiges}

\section{Anmeldung}

Üblicherweise melden wir eine Arbeit erst an,
wenn der Student mit dem Schreiben begonnen hat,
also nach der Implementierung.
Das verringert die Bürokratie und den Stress,
der mit verpassten Deadlines kommt.

Außerdem ist ein Abbrechen nach der Anmeldung ein offizieller Akt
für den es wiederum Fristen gibt:

\begin{center}
\begin{tabular}{lrrr}
\toprule
 & Abbruchfrist nach Anmeldung \\
\midrule
Bachelor      & 4 Wochen \\
Master        & 2 Monate \\
Diplom        & 3 Monate \\
\bottomrule
\end{tabular}
\end{center}

Nach dieser Frist muss die abgebrochene Arbeit mit 5,0 bewertet werden.

Das ISS empfiehlt, dass Studenten sich zusätzlich selbst im Studienportal anmelden.
Das könnte die Eintragung der Note beschleunigen.

\section{Antrittsvortrag}

Bei internen Arbeiten jeglicher Art ist ein Antrittsvortrag optional.
Bei externen Arbeiten ist ein Antrittsvortrag Pflicht.

Dauer: 15 Minuten + 5 Minuten Fragezeit.

Ein Antrittsvortrag sollte nach der Einarbeitungsphase stattfinden,
wenn man einen Überblick hat und weiß was man vorhat.
Im Antrittsvortrag kann man abtasten was Prof.~Snelting von dem Thema hält
und wo man Schwerpunkte setzen oder erweitern sollte.

\section{Abgabe}

\begin{center}
\begin{tabular}{lrrr}
\toprule
 & Dauer & Umfang \\
\midrule
Bachelor      & 4 Monate & 30+ Seiten \\
Master        & 6 Monate & 50+ Seiten \\
Studienarbeit & 3 Monate & 30+ Seiten \\
Diplom        & 6 Monate & 50+ Seiten \\
\bottomrule
\end{tabular}
\end{center}

Man kann eine "4.0 Bescheinigung" bekommen,
bspw.\ für die Masteranmeldungen.

Abzugeben sind jeweils 4 gedruckte Examplare der Arbeit,
das Dokument als pdf Datei
und entstandener Code und andere Artefakte.
Außerdem könnten spätere Studenten dankbar sein für \TeX-Sourcen.

Zum Drucken empfehlen wir
Katz Copy\footnote{\url{http://www.katz-copy.com/}} am Kronenplatz,
weil wir in Sachen Qualität dort die besten Erfahrungen gemacht haben.
Bitte keine Spiralbindung,
da sich das schlecht Stapeln lässt.
Farbdruck ist nicht verpflichtend,
solange in Schwarzweiß noch alle Grafiken lesbar sind.

\section{Abschlussvortrag}

Die Abschlusspräsentation dauert für Bachelorarbeiten 15 Minuten
zuzüglich mind. 10 Minuten für Fragen.
Bei Masterarbeiten sind 20--25 Minuten für den Vortrag vorgesehen.

Der Vortrag soll innerhalb von vier Wochen nach Abgabe erfolgen,
entsprechend Prüfungsordnung.
Die Arbeit muss mindestens einen Tag vor dem Abschlussvortrag abgegeben sein,
damit sich Prof.~Snelting vorbereiten kann.

Am besten direkt im Anschluß den Vortrag ausarbeiten und ein oder zwei Wochen nach Abgabe halten.
Der Präsentationstermin muss ein bis zwei Monate im Voraus geplant werden,
denn Prof.~Snelting hat üblicherweise einen vollen Terminkalender.

\section{Gutachten}

Der Prüfer erstellt ein Gutachten zur Arbeit.
Um das Gutachten einzusehen muss ein Antrag beim Prüfungsamt gestellt werden.
Der Betreuer bzw. Prüfer darf das Gutachten nur mit genehmigtem Gutachten zeigen.
Mündliche Auskunft zur Note ist allerdings möglich.

\section{Bewertung}

\begin{itemize}
  \item Diplom- und Masterarbeiten \emph{müssen} eine wissenschaftliche Komponente enthalten.
    Bachelorarbeit \emph{sollten}, aber zum Bestehen ist es nicht notwendig.
    Wissenschaftlich ist was über reine Implementierungs- bzw. Softwareentwicklungsaufgaben hinausgeht.
    Üblicherweise findet man theoretische Betrachtungen zu Korrektheit und Effizienz.
    Willkürliche Daumenregel: Ohne Formel, keine Wissenschaft.
  \item Diplom- und Masterarbeiten benötigen eigentlich immer Wissen aus dem Diplom- bzw. Masterstudium.
    Falls das Wissen aus Vordiplom bzw. Bachelor ausreicht,
    sollte man nochmal darüber nachdenken.
  \item Positiv mit der Note korrelieren
    selbstständiges Arbeiten,
    regelmäßige Abstimmung mit dem Betreuer,
    mehrere Feedbackrunden mit verschiedenen Leuten,
    mehrmaliges Üben des Abschlussvortrags,
    Einbringen eigener Ideen,
    gutes Zuhören
    und sorgfältiges Debugging.
  \item Negativ mit der Note korrelieren
    wochenlanges Pausieren,
    Ignorieren von Feedback,
    Deadlines überziehen
    und Arbeiten im stillen Kämmerchen.
\end{itemize}

Disclaimer:
Nein, es gibt keinen konkreten Notenschlüssel.
Die obigen Punkte sind nur grobe Richtlinien und für niemanden in irgendeiner Weise bindend.

\section{\LaTeX\ Features}

\subsection{Schriftformatierungen}

\begin{tabular}{lccc}
\toprule
 & serif & sans-serif & fixed-width \\
\midrule
normal        & \textrm{\textup{Medium}} \textrm{\textup{\textbf{Bold}}} & \textsf{\textup{Medium}} \textsf{\textup{\textbf{Bold}}} & \texttt{\textup{Medium}} \texttt{\textup{\textbf{Bold}}} \\
italic        & \textrm{\textit{Medium}} \textrm{\textit{\textbf{Bold}}} & \textsf{\textit{Medium}} \textsf{\textit{\textbf{Bold}}} & \texttt{\textit{Medium}} \texttt{\textit{\textbf{Bold}}}\\
slanted       & \textrm{\textsl{Medium}} \textrm{\textsl{Bold}} & \textsf{\textsl{Medium}} \textsf{\textsl{\textbf{Bold}}} & \texttt{\textsl{Medium}} \texttt{\textsl{\textbf{Bold}}} \\
small-capital & \textrm{\textsc{Medium}} \textrm{\textsc{\textbf{Bold}}} & \textsf{\textsc{Medium}} \textsf{\textsc{\textbf{Bold}}} & \texttt{\textsc{Medium}} \texttt{\textsc{\textbf{Bold}}} \\
\bottomrule
\end{tabular}

Math fonts:
$\mathnormal{absXYZ}$,
$\mathrm{absXYZ}$,
$\mathbf{absXYZ}$,
$\mathsf{absXYZ}$,
$\mathit{absXYZ}$,
$\mathtt{absXYZ}$, and
$\mathcal{XYZ}$.

\subsection{Rand und Platz}

Viele Benutzer von \LaTeX\ wollen Ränder und Seitengröße anpassen.
Dazu empfehlen wir erstmal die KOMA Script Dokumentation (\texttt{koma-script.pdf}) zu lesen,
insbesondere Kapitel 2.2.
Bevor man mit \texttt{\textbackslash enlargethispage}
oder ähnlichen Tricks anfängt,
sollte man \texttt{\textbackslash typearea} anpassen.

Falls die Arbeit auf Englisch verfasst wird,
sollte man wissen, dass Absätze im Englischen üblicherweise anders formatiert werden.
Im Deutschen macht man eine Leerzeile zwischen Absätzen.
Im Englischen wird stattdessen die erste Zeile eines Absatzes eingerückt.
