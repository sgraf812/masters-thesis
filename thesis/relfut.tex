\chapter{Related and Future Work}\label{sec:relfut}

This section is dedicated to comparison of our analysis with prior approaches in \cref{sec:rel} and what loose ends can be tied up in the future in \cref{sec:fut}.

\section{Related Work}\label{sec:rel}

\subsubsection{Abstract Interpretation}

Within the framework of \emph{abstract interpretation}, questions of cardinality have been successfully answered by backwards analyses based on \emph{projections} \parencite{projs} in the past.

Being an extension to the approach of \textcite{card}, with all the bells and whistles such as call and product uses, our usage analysis is no different:
What we called usage transformers came into the world as the strictness-specific concept of \emph{projection transformers} \parencite{projimpl}, describing how a use on an expression translates into a use on its free variables and arguments.
We showed how the overlap in Call Arity \parencite{callarity} and Cardinality Analysis (as in \parencite{card}) can be leveraged by giving a usage analysis that generalises the results of both analyses, in order to possibly replace both in the long run.

Key was the observation that call arity (the minimum number of arguments a binding is applied to) can be computed independently of whether $\eta$-expanding the binding to that arity is possible, considering sharing.
In other words:
By enriching the domain of discourse to model more precise usage information, we could break the interleaving of sharing and arity analysis out of Call Arity \parencite{callarity}.
Usage information is now computed by our usage analysis, while the arity analysis is done by GHC's regular arity analysis, specifically feeding on one-shot annotations\footnote{Provided a little incentive from our side to always $\eta$-expand cheap (according to GHC's cost model) expressions, so that the behavior of Call Arity is matched.}.

\subsubsection{Type Systems}

For comparing our analysis to approaches based on type systems, we kindly refer to \textcite[Section~8]{card}, who provide a great overview over recent advances.
We provide a summary of their excellent writing for completenes.

While linear type systems can express single-entry and one-shot annotations quite naturally (modulo the caller \vs callee distiction \parencite{card}), they proved to be too restrictive \parencite{polytype}.
Nonetheless, this led to a series of papers on type systems, specifically tailored to do usage analysis \parencite{type}.
\parencite{polytype} identified polymorphism and subtyping to be essential for good results, which in conjunction with an exponentially growing number of annotations needed \parencite{warnsbrough} elicited unacceptable complexity, both in terms of performance and in the implementation.

The main disadvantage of the approach by abstract interpretation is worse approximations across function boundaries.
\textcite{card} (and therefore our solution) recovers a good deed of that opportunity by computing usage signatures to appropriately analyse first- and second-order functions.
Also, the very aggressive inliner of GHC reduces the cases where interprocedural information flow is important to recursive functions.

\textcite{card} also compare the analysis strategy for \hsinl{let} bindings to that of type system.
They conclude that type systems operate similar to \letupsc, which has several drawbacks compared to \letdnsc (corresponding to the more operational view) as outlined in \cref{sec:let}.
Dealing with free variables in a precise manner requires polymorphic effect systems as in \textcite{sharing}, which try to alleviate some of the pain regarding subtyping.

A recent type-based approach by \textcite{verstoep} seemingly originating from his master's thesis \parencite{verstoepthesis}. 
It offers a similar take at the problem as the Demand Analyser, computing all relevant cardinality information (absence, sharing, strictness, uniquess) in one run.
Strongly inspired by \textcite{warnsbrough}, they differentiate \emph{use} from \emph{demand}, resp. \emph{call use} and \emph{evaluation} in our language (\cf \cref{sec:zoo}), in order to handle \hsinl{seq} appropriately.
They plan to integrate their work into the Utrecht Haskell Compiler (UHC), the inliner of which doesn't seem to be as aggressive as GHC's.
According to own claims \parencite{verstoepthesis}, the approach enriches the work of \textcite{warnsbrough} with a combination of polymorphism and polyvariance, also adding an option for uniqueness typing.
Although the ensemble of polymorphism, subtyping and annotating data types was identified as problematic by \textcite{card}, it is argued that running the analysis plays out in terms of compiler performance, because it provides a wealth of information \parencite{verstoep}.
There are no numbers yet to substantiate that claim, appearently because integration into UHC isn't finished yet.

\subsubsection{Data-flow Frameworks}

For the implementation of imperative languages, data-flow analysis on control flow graphs or directly on graph-based intermediate representations \parencite{firm} \parencite{thorin} are the norm.

There's even a Haskell library for analysing and transforming control flow graphs, \texttt{hoopl}, introduced in \textcite{hoopl} and used in the C-- backend of GHC.
On first sight, our \texttt{Worklist} module seems to be a direct contender to \texttt{hoopl} and in fact we considered to use \texttt{hoopl} for our purposes.
However, as we discovered the API, it became evident quickly that expressing our rather complex analysis in terms of the traditional gen/kill set terminology was quite impossible. 

Thus we created our own solution to the problem, which worked out quite well for us:
We got to abstract data-flow iteration logic behind a \hsinl{State} monad, for which we exposed a single impure primitive, leading to analysis logic that is completely decoupled from fixed-point iteration and change detection logic.

\section{Future Work}\label{sec:fut}

Developing the deep understanding of Call Arity and Cardinality Analysis necessary to merge the two in itself was quite time consuming.
Implementing a few unmentioned iterations of the analysis, combined with fixing all the repercussions of hacking on a central part of GHC took even more time.
Also considering the effort invested in fleshing out the writing, it is clear that some things were out of scope for this thesis.

For reasons conjectured in \cref{sec:bench}, the full analysis can't cope with some input programs, space-wise.
Interestingly, the problem wasn't so much the topology of graphs (which were almost exclusively sparse or dense), so it remains to be seen if the painful resource usage explosion can be remedied.
The obvious solution is to employ the much better performing third variant, which still provides a benefit over the separate Call Arity and Cardinality Analysis at hardly any cost in precision.

It would be interesting to see if some more explicitly operational model of sharing would alleviate the need for the \letupsc analysis order, as it is responsible for some imprecision even when refined with co-call graphs (\cf the examples in \cref{sec:let}).
All our attempts to split off shared evaluation failed because of the complex interaction with sequential composition (\eg $\both$).
Also, the grain of improvement to precision probably wouldn't matter much, as we already are at the far end of diminishing returns, it seems.
But it would be interesting to see the comparison to variant \varedges exactly for that reason.

The requirement for an iteration order decoupled from the syntax tree came from better arity-aware \letdnsc-style analysis.
In \cref{sec:solve}, we quickly came to the conclusion that in order to keep our sanity, we had to separate iteration logic from analysis logic.

This turned out to be an interesting path that no-one seems to have taken before:
We abstract in the \hsinl{Worklist} module a (currently rather limited) approach to iterating data-flow frameworks defined by mutually recursive \hsinl{TransferFunction}s.
Our formulation still misses a lot of potential abstraction and performance tuning.

Thinking in broader strokes, we could easily see this graph-based fixed-point iteration scheme become more widely used throughout GHC's analyses.
In order for that to happen, assignment of \hsinl{FrameworkNode}s to syntactic elements should be done once in a central place after transformation passes, that this needless redundancy won't bleed into analysis logic.

There's also the issue of monotonicity, discussed in \cref{sec:mono}.
This is a consequence of how we encoded usage transformers:
The lack of an appropriate data structure to model monotone maps over join-semilattices forced us to use ordinary maps instead.
The total ordering used for the balanced search tree doesn't coincide with the join-semilattice structure expressed by $\llub$ in any way, so the potentially helpful functions \hsinl{Map.lookupLT} were of no use at all. 
Let alone that they have the wrong return type to express multiple lower bounds, which are possible in a lattice.
Thus, an easy improvement over the current situation would be to look out for an algorithmic solution to lookup values in a data structure that is keyed by a (join semi-)lattice.
The underlying data structure would have to maintain multiple sources of a directed acylclic graph to be remotely efficient, but considering that only very few points of a usage transformer are ever requested, this seems like no big deal.

A data structure for monotone maps would completely get rid of the monotonicity hacks in \cref{sec:mono} and conceivably lead to better performance of the analysis overall.

Strictness analysis possibly benefits from the same improvements to \letdnsc analysis strategy, which would be interesting to pursue.
Shattering the Demand Analyser in its three parts (usage, strictness, constructed product result analysis) sketched in \cref{sec:untangle} seems worthwhile from a perspective of software engineering and could also lay the ground work for integrating the graph-based data-flow iteration approach.

Lastly, we feel like there needs to be held a discussion about what the arity analyser should compute.
Some GHC comments contradict each other in the details of what \hsinl{idArity} is really supposed to mean:
There's the notion of `does essentially no work until applied to \hsinl{idArity} arguments' (in the haddock of \hsinl{ArityInfo}) \vs `We want expressions returning one-shot lambdas to have arity one' (Note `One-shot lambdas' in \hsinl{CoreArity}).
Either approach seems fine to us, but documentation should be clear about it, and possibly integrate usage information directly instead of one-shot annotations only.
We anticipate some advantages in looking at the usage of a binder:
The example in Note `Arity analysis' in \hsinl{CoreArity} would not need any fixed-pointing to figure out that \hsinl{f} has arity 2; this property seems entirely specific to how many arguments \hsinl{f} is applied to within its scope, a typical usage property.
E.g., a usage analysis would compute the dreaded fixed-point to find out a usage of $\omega*C^\omega(C^1(U))$, based on which the arity analyser could compute the arity based on the arity expansion procedure in \cref{sec:expand}.
Of course, even without messing about with usage information, the arity analyser aggregates more than enough heuristic concern (\eg if an \hsinl{exprIsCheap} enough to arity expand over) to justify its existence.

Hence, cutting out any logic in the arity analyser that computes usage information seems like a good idea to reduce intellectual and runtime cost of the analysis (though it's arguably rather cheap anyway).
