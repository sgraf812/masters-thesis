\chapter{Conclusion}\label{sec:conclusion}

Demand Analysis and Call Arity share subtle commonalities.
Our work made these very explicit by providing a usage analysis that subsumes Call Arity and the usage analysis within Demand Analysis, referred to as Cardinality Analysis.

We highlighted problematic differences between the two (\cref{sec:intro}), like the \letupsc \vs \letdnsc analysis order, and showed how to systematically solve them.

In \cref{sec:untangle}, we advocated a clear separation of concerns and a split of the Demand Analyser in its sub analyses, consequently.
We substantiated our claim by pointing out a number of tradeoffs associated with performing all analyses in lockstep, without any real data dependency between them which justifies doing so.

Although the abstract domain in which we interpret programs is not fundamentally different to that of \textcite{card}, we gave a more precise meaning to each element of the domain.
In particular our treatment of usage transformers was more explicit than in \textcite{card}, which was needed because we denote our programs with them via our transfer function in \cref{sec:transfer}.

As \cref{sec:impl} reveals, implementing the analysis was a particular challenges, as we needed to provide quite some infrastructure before writing the analysis in the final form was even feasible.
We are particularly proud of our take on solving data-flow problems with a graph-based approach, decoupled from the syntax tree in \cref{sec:solve} and expect interesting further discussions.
There were quite some hacks involved in making the analysis realisable, some of which evoked further problems, like the lack of a data structure for maps indexed by join-semilattices in \cref{sec:mono}.

Although according to \cref{sec:eval} the full generalisation has performance issues due to ubiquotious use of co-call graphs (which might be entirely related to the implementation), a variant without co-call graphs seems to compromise little on precision but still reaches performance comparable to prior approaches.

Opportunities for future work were discussed in \cref{sec:fut} and left the author excited to pursue some open problems and to contribute to GHC in the future.
